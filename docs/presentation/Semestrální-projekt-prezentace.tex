\documentclass[lualatex,hyperref={pdfencoding=auto}]{beamer}
\usepackage[czech]{babel}
% \usepackage[utf8]{inputenc}
\usepackage{biblatex}
\addbibresource{citace.bib}
\usetheme[fei]{vsb}



\usepackage{tikz}
\usetikzlibrary{positioning,calc}

% \bibliography{citace.bib}

% \addbibresource{citace.bib}

\title[Komprese stromových struktur]{Komprese stromových struktur}
\subtitle{Semestrální projekt}
\author{Marek Beran}
\institute[VŠB-TUO]{VŠB -- Technická univerzita Ostrava\\\vspace{2mm}marek.beran.st@vsb.cz}
\date[23.~5.~2025]{23.~května 2025}

\showboxdepth=5

\begin{document}

\section{Úvod}

\begin{frame}{Obsah}
    \tableofcontents
\end{frame}

\begin{frame}{Cíl a motivace}

\begin{itemize}
    \item Je možné efektivně komprimovat textové data převedením do stromové struktury?
    \item Cíl: Proof of Concept
\end{itemize}
\end{frame}

\section{Implementace knihovny}
\begin{frame}{Použité technologie}
    \begin{columns}[t] 
        \begin{column}{0.5\textwidth}
            \vspace{0pt} % Vynucení zarovnání nahoře
            \textbf{Programovací jazyk a platforma:}
            \begin{itemize}
                \item C\# 9.0
                \item .NET 5.0 a vyšší
                \item Visual Studio 2022
            \end{itemize}
            \vspace{12pt}
            \textbf{Knihovny:}
            \begin{itemize}
                \item UDPipe 
                (rozpoznávání syntaktických stromů) \nocite{udpipe:2017}
                \item MorphoDiTa (morfologická analýza) \nocite{strakova14}
            \end{itemize}
        \end{column}
        \begin{column}{0.5\textwidth}
            \vspace{0pt} % Vynucení zarovnání nahoře
            \textbf{Další nástroje:}
            \begin{itemize}
                \item R (datová analýza a vizualizace)
                \item Mkdocs (dokumentace)
                \item Bash skripty (podpůrné nástroje)
            \end{itemize}
            
            \textbf{Bindings:}
            \begin{itemize}
                \item C\# wrapper pro UDPipe (nativní knihovna) 
                \item C\# wrapper pro MorphoDiTa (nativní knihovna) 
            \end{itemize}
        \end{column}
    \end{columns}
\end{frame}


\begin{frame}{Implementace knihovny}
  \nocite{Oquendo2016Pipe}
    \begin{figure}
        \centering
        \includegraphics[width=\textwidth]{fig/class-diagram.png}
        \caption{Třídní diagram části implementace zaměření na řetězení filtrů}
        \label{fig:class-diagram}
    \end{figure}
\end{frame}

\section{Převod textu do stromové struktury}

\begin{frame}{Převod textu do stromové struktury}
  \begin{itemize}
    \item Dependency parsing - závislosti mezi slovy ve větě.\nocite{kubler2009dependency} \nocite{jurafsky2024speech} \nocite{filippova2008dependency}
    \item Využití knihovny UDPipe pro syntaktickou analýzu textu
    \item Vytvoření syntaktického stromu pro každou větu
  \end{itemize}
  \centering
  \begin{tikzpicture}[
% Node styles
 box/.style={draw, rounded corners, fill=blue!10, font=\sffamily,
 minimum width=1.2cm, minimum height=0.8cm, align=center},
 root/.style={fill=blue!20},
% Edge styles
 edge/.style={draw, thick, black!50},
% Label styles
 label/.style={midway, font=\small\sffamily}
]
% Root node
\node[box, root] (jumps) at (0,0) {jumps};
% Left branch (fox and children)
\node[box] (fox) at (-2,-1) {fox};
\node[box] (the) at (-6.6,-3) {The};
\node[box] (quick) at (-4.6,-3) {quick};
\node[box] (brown) at (-2.6,-3) {brown};
% Right branch (over and descendants)
\node[box] (dog) at (3,-3) {dog};
\node[box] (over) at (-2,-5) {over};
\node[box] (the2) at (0,-5) {the};
\node[box] (lazy) at (2,-5) {lazy};
% Connect nodes with edges
\draw[edge] (jumps) -- (fox) node[label, left] {nsubj};
\draw[edge] (fox) -- (the) node[label, left] {det};
\draw[edge] (fox) -- (quick) node[label, left] {amod};
\draw[edge] (fox) -- (brown) node[label, right] {amod};
\draw[edge] (jumps) -- (dog) node[label, right] {case};
\draw[edge] (dog) -- (over) node[label, left] {obl};
\draw[edge] (dog) -- (the2) node[label, left] {det};
\draw[edge] (dog) -- (lazy) node[label, right] {amod};
\end{tikzpicture}
\end{frame}

\begin{frame}{Umělé rozšíření stromu}
  \begin{itemize}
    \item Rozšíření stromu pro podporu více vět bez nutnosti práce s lesem
    \item Pro zajištění dostatečné velikosti syntaktického stromu pro testování a kompresi
  \end{itemize}
  \centering
    \begin{tikzpicture}[
% Node styles
 box/.style={draw, rounded corners, fill=blue!10, font=\sffamily,
 minimum width=1.2cm, minimum height=0.8cm, align=center},
 root/.style={fill=blue!20},
% Edge styles
 edge/.style={draw, thick, black!50},
% Label styles
 label/.style={midway, font=\small\sffamily}
]
\node[box] (document) at (0,0) {<Document>};

\node [box] (root1) at (-3,-3) {<Root>};
\node [box] (root2) at (-1,-3) {<Root>};
\node [box] (root3) at (1,-3) {<Root>};
\node [box] (root4) at (3,-3) {<Root>};

\draw[edge] (document) -- (root1);
\draw[edge] (document) -- (root2);
\draw[edge] (document) -- (root3);
\draw[edge] (document) -- (root4);

\draw[edge, opacity=0.3] (document) -- ++(-5, -2.5);
\draw[edge, opacity=0.3] (document) -- ++(5, -2.5);

\draw[edge, opacity=0.3] (root1) -- ++(0, -1);
\draw[edge, opacity=0.3] (root2) -- ++(0, -1);
\draw[edge, opacity=0.3] (root3) -- ++(0, -1);
\draw[edge, opacity=0.3] (root4) -- ++(0, -1);

\node at (-5,-3) {\Huge$\cdots$};
\node at (5,-3) {\Huge$\cdots$};


\node at (-3,-4.2) {\Huge$\vdots$};
\node at (-1,-4.2) {\Huge$\vdots$};
\node at (1,-4.2) {\Huge$\vdots$};
\node at (3,-4.2) {\Huge$\vdots$};

\end{tikzpicture}
\end{frame}

\section{Algoritmy}
\begin{frame}{Algoritmy}
  \begin{itemize}
    \item Zaměření na gramatickou kompresi
    \item Komprimace pomocí linearizace -- převod stromu na posloupnost uzlů a jejich následná komprese pomocí algoritmů pro kompresi textu \nocite{McAnlis2016}
  \end{itemize}
\end{frame}

\begin{frame}{Linearizace}
  \begin{itemize}
    \item Defakto převod stromu zpět na textovou reprezentaci
    \item Zvýšená redundance v důsledku zachování stromové struktury
    \item Řešeno hloubkovým průchodem -- dosahoval nejlepších výsledků    
  \end{itemize}
\end{frame}

% \begin{frame}
%   \begin{itemize}
    
%   \end{itemize}
% \end{frame}

\section{Literatura}
\begin{frame}[allowframebreaks]
    \frametitle{Literatura}
    \printbibliography[heading=none]
\end{frame}

\end{document}